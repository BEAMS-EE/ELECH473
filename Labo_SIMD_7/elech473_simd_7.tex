\documentclass[10pt,a4paper]{article}
\usepackage[utf8x]{inputenc}
\usepackage[T1]{fontenc}
%\usepackage{stringenc} % for grffile
\usepackage{ucs}
\usepackage{amsthm} %numéroter les questions
\usepackage[english]{babel}
\usepackage{datetime}
\usepackage{xspace} % typographie IN
\usepackage{hyperref}% hyperliens
\usepackage[all]{hypcap} %lien pointe en haut des figures
\usepackage[english]{varioref} %voir x p y
\usepackage{fancyhdr}% en têtes
%\input cyracc.def
\usepackage[]{graphicx} %include pictures
%\usepackage[encoding,inputencoding=utf8,filenameencoding=utf8]{grffile}
%\usepackage[extendedchars,inputencoding=latin1,filenameencoding=latin1]{grffile}
\usepackage[siunitx ]{circuitikz}
%\usepackage{gnuplottex}
\usepackage{ifthen}
\graphicspath{{./figures/}}
\usepackage{array}
\usepackage{amsmath}
\usepackage[]{xcolor}
\usepackage{tikz}
\usepackage{tikz-timing}
\usetikzlibrary{scopes}
\usetikzlibrary{backgrounds}
\usepackage{listings}
\usepackage{enumitem}
\usepackage[top=1 in, bottom=1 in, left=1.3 in, right=1 in]{geometry} % Yeah, that's bad to play with margins
\usepackage[]{pdfpages}
\usepackage{pdflscape}
\usepackage[]{attachfile}
\usepackage{colortbl}
\usepackage{multirow}
\usepackage{booktabs}
\usepackage{rotating}
\usepackage{fontawesome} %Symbols
\usepackage[cache=false]{minted}

\newcommand{\version}{v1.0.0}

%cyr
%\newcommand\textcyr[1]{{\fontencoding{OT2}\fontfamily{wncyr}\selectfont #1}}


\newboolean{corrige}
%\setboolean{corrige}{true}%corrigé
\setboolean{corrige}{false}% pas de corrigé

\usepackage{aeguill} %guillemets

%% fancy header & foot
\pagestyle{fancy}
\lhead{[ELEC-H-473] Microprocessor Architectures: SIMD 1}
\rhead{\version\\ page \thepage}
\chead{\ifthenelse{\boolean{corrige}}{Corrigé}{}}
\cfoot{}
%%

\pdfinfo{
/Author (La saucisse électronique volante)
/Title (ELECH473 Lab 7 - SIMD 1)
/ModDate (D:\pdfdate)
}

\hypersetup{
pdftitle={ELECH473 Lab 7 - SIMD 1},
pdfauthor={Mouette unijambiste vengeresse},
pdfsubject={SIMD 1}
}

\theoremstyle{definition}% questions pas en italique
\newtheorem{Q}{Question}[] % numéroter les questions [section] ou non []

\newcommand{\reponse}[1]{% pour intégrer une réponse : \reponse{texte} : sera inclus si \boolean{corrige}
	\ifthenelse {\boolean{corrige}} {\paragraph{Réponse :} #1} {}
 }

\newcommand{\addcontentslinenono}[4]{\addtocontents{#1}{\protect\contentsline{#2}{#3}{#4}{}}}

\newcommand{\on}[1]{{\operatorname{#1}}}

\def\labelitemi{--}
\setlist{parsep=0pt,itemsep=0pt,style=standard,leftmargin=\parindent, align=left} % pas d'espace prohibitif entre les items
\setlist{nolistsep}

\newcolumntype{C}[1]{>{\centering\let\newline\\\arraybackslash\hspace{0pt}}m{#1}}

\setlength{\tabcolsep}{0pt} %no extra space in cells to keep constant tabular width

\date{\vspace{-1cm}\version}
\title{\vspace{-2cm} Lab 7\\ Microprocessor Architectures [ELEC-H-473]\\ SIMD 1: Image threshold\ifthenelse{\boolean{corrige}}{~\\Corrigé}{}}

%\author{\vspace{-1cm}}%\textsc{Yannick Allard}}


\lstdefinestyle{customasm}{
 % belowcaptionskip=1\baselineskip,
 % frame=L,
 % xleftmargin=\parindent,
 language=[x86masm]Assembler,
 basicstyle=\footnotesize\ttfamily,
 commentstyle=\itshape\color{magenta!40!black},
  comment=[l]//,
}

\lstset{escapechar=@,style=customasm}

\begin{document}

% Introduce a new counter for counting the nodes needed for circling
\newcounter{nodecount}
% Command for making a new node and naming it according to the nodecount counter
\newcommand\tabnode[1]{\addtocounter{nodecount}{1} \tikz \node (\arabic{nodecount}) {#1};}

% Some options common to all the nodes and paths
\tikzstyle{every picture}+=[remember picture,baseline]
\tikzstyle{every node}+=[inner sep=0pt,anchor=base]
\tikzstyle{every path}+=[thick, rounded corners]



\maketitle
\section*{Description of the lab}
Objetives:
\begin{itemize}
	\item Learn how to use Code::Blocks.
	\item Write	elementary C programs and how to use SIMD with inline assembly.
	\item Learn how to use the debugging environment.
	\item For the proposed problem, implement C and SIMD versions of the program.
	\item Measure execution times to compare different implementations and understand the advantages of SIMD.
\end{itemize}

\section*{Preliminary}
\begin{itemize}
	\item Start by writing a ``Hello World'' program.
	\item Modify your code by adding some arithmetical operations.
	\item Launch the program in debug mode and execute it step-by-step.
	\item Learn how to observe register file change.
\end{itemize}

\section*{Turning a grey scale image black and white}
Digital images can be represented in the form of a two-dimensional matrix where indexes i,j represent the spatial coordinates of the image.
For a colour image there are three matrices, one per colour component R, G and B (Red, Green, Blue).
For a grey level image, only one matrix is necessary (R, G, B components have the same value).
Depending on image quantification, we can have more or less colours.
Most of the time grey level images are coded using 8 bits of information per pixel.
This means 256 colours with:
\begin{itemize}
	\item Black is $(0)_{10} = (00000000)_2$
	\item White is $(255)_{10} = (11111111)_2$
\end{itemize}

In image processing applications, it is often useful to work with binary images that use only black and white colours.
For such images, the pixel value is given only with a binary choice: black (0) or white (1).
For images coded with 8 bits, this means 0 or 255 (0x00 or 0xFF in hexadecimal).
In order to convert one grey level image into a black and white binary image we need to fix a threshold value.
If the current pixel value is smaller than the threshold value, we will set the new pixel value to 0; otherwise it will be set to 255.
Depending on the input image and the value of the threshold we will obtain different resulting images.
For this exercise you can use the value of the threshold as you like, we are only interested in the way computation is being performed (and the execution time).

Here is the pseudo-code of what you will have to program:
\begin{minted}{C}
for (all_images) {
	read_image_from_file
	start_time = get_time ()
	for (all_pixels_in_the_image) {
		if (curent_pixel_value < treshold) then new_pixel_value = 0
		else new_pixel_value = 255 }
	end_time = get_time ()
	out >> "Computing time " end_time - start_time
	write_image_result
}
\end{minted}

To do so, you will need to include assembly inside your C code:

\begin{minted}{C}
fonctionC()
{
	_asm{
		/* ASM code */
		"sub $1, %%rdx\n"
	}
}
\end{minted}

\section*{Assignment}
\begin{enumerate}
	\item Implement the image threshold processing in plain C.
	\item Implement the same algorithm in SIMD.
	\item Benchmark both solutions and compare their performance.
	If the precision of your benchmarking is not sufficient (especially on Windows), run the filter several times (only the filter, not the file reading and writing).

	Comment on the expected acceleration factor, the actual results, the difference between debug and release compilation, etc.
\end{enumerate}

\subsection*{Code requirements}

\begin{itemize}
	\item Send only your source files, \textit{i.e.} .c/.h/.cpp/.hpp. Do \textit{not} send the input and output files.
	\item If your code requires specific configuration flags for the compilation, please write them down as comments in your code.
	\item Make sure it's well documented.
	\item All the input and output files should be in the source directory.
	\item All the output files should have the same base name as the source files on the UV (ie. Escher, kid or lena\_gray), appended with ``out\_C.raw'' or ``out\_SIMD.raw''.
	\item Do not use any weird library that may not compile on a Linux-based computer.
\end{itemize}


\end{document}
