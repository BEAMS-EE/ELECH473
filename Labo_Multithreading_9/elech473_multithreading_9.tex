\documentclass[10pt,a4paper]{article}
\usepackage[utf8x]{inputenc}
\usepackage[T1]{fontenc}
%\usepackage{stringenc} % for grffile
\usepackage{ucs}
\usepackage{amsthm} %numéroter les questions
\usepackage[english]{babel}
\usepackage{datetime}
\usepackage{xspace} % typographie IN
\usepackage{hyperref}% hyperliens
\usepackage[all]{hypcap} %lien pointe en haut des figures
\usepackage[english]{varioref} %voir x p y
\usepackage{fancyhdr}% en têtes
%\input cyracc.def
\usepackage[]{graphicx} %include pictures
%\usepackage[encoding,inputencoding=utf8,filenameencoding=utf8]{grffile}
%\usepackage[extendedchars,inputencoding=latin1,filenameencoding=latin1]{grffile}
\usepackage[siunitx ]{circuitikz}
%\usepackage{gnuplottex}
\usepackage{ifthen}
\graphicspath{{./figures/}{./images/}}
\usepackage{array}
\usepackage{amsmath}
\usepackage[]{xcolor}
\usepackage{tikz}
\usepackage{tikz-timing}
\usetikzlibrary{scopes}
\usetikzlibrary{backgrounds}
\usepackage{listings}
\usepackage{enumitem}
\usepackage[top=1 in, bottom=1 in, left=1.3 in, right=1 in]{geometry} % Yeah, that's bad to play with margins
\usepackage[]{pdfpages}
\usepackage{pdflscape}
\usepackage[]{attachfile}
\usepackage{colortbl}
\usepackage{multirow}
\usepackage{booktabs}
\usepackage{rotating}
\usepackage{fontawesome} %Symbols
\usepackage{subcaption}
\usepackage{minted}


\newcommand{\version}{v1.0.1}

%cyr
%\newcommand\textcyr[1]{{\fontencoding{OT2}\fontfamily{wncyr}\selectfont #1}}


\newboolean{corrige}
%\setboolean{corrige}{true}%corrigé
\setboolean{corrige}{false}% pas de corrigé

\usepackage{aeguill} %guillemets

%% fancy header & foot
\pagestyle{fancy}
\lhead{[ELEC-H-473] Microprocessor Architectures: Multithreading}
\rhead{\version\\ page \thepage}
\chead{\ifthenelse{\boolean{corrige}}{Corrigé}{}}
\cfoot{}
%%

\pdfinfo{
/Author (La saucisse électronique volante)
/Title (ELECH473 Lab 9 - Multithreading)
/ModDate (D:\pdfdate)
}

\hypersetup{
pdftitle={ELECH473 Lab 9 - Multithreading},
pdfauthor={Mouette unijambiste vengeresse},
pdfsubject={SIMD 1}
}

\theoremstyle{definition}% questions pas en italique
\newtheorem{Q}{Question}[] % numéroter les questions [section] ou non []

\newcommand{\reponse}[1]{% pour intégrer une réponse : \reponse{texte} : sera inclus si \boolean{corrige}
	\ifthenelse {\boolean{corrige}} {\paragraph{Réponse :} #1} {}
 }

\newcommand{\addcontentslinenono}[4]{\addtocontents{#1}{\protect\contentsline{#2}{#3}{#4}{}}}

\newcommand{\on}[1]{{\operatorname{#1}}}

\def\labelitemi{--}
\setlist{parsep=0pt,itemsep=0pt,style=standard,leftmargin=\parindent, align=left} % pas d'espace prohibitif entre les items
\setlist{nolistsep}

\newcolumntype{C}[1]{>{\centering\let\newline\\\arraybackslash\hspace{0pt}}m{#1}}

\setlength{\tabcolsep}{0pt} %no extra space in cells to keep constant tabular width

\date{\vspace{-1cm}\version}
\title{\vspace{-2cm} Lab 9\\ Microprocessor Architectures [ELEC-H-473]\\ Multithreading \ifthenelse{\boolean{corrige}}{~\\Corrigé}{}}

%\author{\vspace{-1cm}}%\textsc{Yannick Allard}}


\lstdefinestyle{customasm}{
 % belowcaptionskip=1\baselineskip,
 % frame=L,
 % xleftmargin=\parindent,
 language=[x86masm]Assembler,
 basicstyle=\footnotesize\ttfamily,
 commentstyle=\itshape\color{magenta!40!black},
 comment=[l]//,
}

\lstset{escapechar=@,style=customasm}

\begin{document}

% Introduce a new counter for counting the nodes needed for circling
\newcounter{nodecount}
% Command for making a new node and naming it according to the nodecount counter
\newcommand\tabnode[1]{\addtocounter{nodecount}{1} \tikz \node (\arabic{nodecount}) {#1};}

% Some options common to all the nodes and paths
\tikzstyle{every picture}+=[remember picture,baseline]
\tikzstyle{every node}+=[inner sep=0pt,anchor=base]
\tikzstyle{every path}+=[thick, rounded corners]



\maketitle
\section*{Description of the lab}
Objectives:
\begin{itemize}
 \item Learn how to create multiple threads.
 \item Quantify potential acceleration.
 \item Understand the limits of acceleration due to memory bottlenecks.
\end{itemize}

\section*{Exercise}
\begin{itemize}
    \item Read and execute the source code provided bellow and understand different concepts implemented.
    The code is also available \href{https://gist.github.com/parastuffs/a7818b1ff46de40b95949943fbd53c8b}{online}.
    \item Understand the thread manipulation functions (thread creation, mutex).
    \item Start from the exercises that you have already did in 2 previous session (use both threshold and max function).
    \item Initiate 2, 4 and 8 threads that work on different image subsets.
    \item Think of different image subset shapes, what would be the ideal one?
\end{itemize}


\begin{minted}{c}
/**
* This needs to be compiled with the pthread library.
* Command line: gcc -lpthread multithread.c
* In CodeBlocks: in the project build options, go in the "Linker settings" tab 
* and add -lpthread in the "Other linker options" area.
**/
#include <stdio.h>
#include <stdlib.h>
#include <pthread.h>

// Define a structure for the thread arguments
typedef struct Args Args;
struct Args {
    int i;
};
 
// Create a global variable
int g = 0;

// Create a global mutex
pthread_mutex_t mutex = PTHREAD_MUTEX_INITIALIZER;
 
// This function is the entry point of the thread
void *myThreadFun(Args *vargp)
{
    // Store the argument passed to this thread
    Args *arg = vargp;
 
    // Create a static variable
    static int s = 0;

    // Normal int variable
    int n = 0;

    ++s;
    pthread_mutex_lock(&mutex);
    ++g;
    pthread_mutex_unlock(&mutex);
    ++n;
 
    printf("Thread ID: %d, Static: %d, Global: %d, Normal: %d\n", arg->i, s, g, n);
    pthread_exit(NULL);
}

int main()
{
    int i;
    int maxThreads = 3;

    // Create thread IDs
    pthread_t *tid = malloc(maxThreads * sizeof(pthread_t));
 
    // Create three threads
    for (i = 0; i < maxThreads; i++) {
        Args *arg = (Args*) malloc(sizeof(Args));
        arg->i = i;
        pthread_create(&tid[i], NULL, myThreadFun, arg);
    }

    // Wait for all the thread to finish
    for (i = 0; i < maxThreads; i++) {
        pthread_join(tid[i], NULL);
    }

    printf("That's it, folks!\n");
    
    return EXIT_SUCCESS;
}
\end{minted}

\end{document}
